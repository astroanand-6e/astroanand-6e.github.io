\documentclass[10pt, letterpaper]{article}

% Packages:
\usepackage[
    top=2cm, bottom=2cm, left=2cm, right=2cm, footskip=1.0cm
]{geometry}
\usepackage{titlesec, tabularx, array, enumitem, fontawesome5, amsmath, hyperref, bookmark, paracol, ifthen, needspace, iftex, lastpage, changepage}
\usepackage[dvipsnames]{xcolor}

% Define Colors
\definecolor{primaryColor}{RGB}{0, 0, 0}

% Metadata
\hypersetup{
    pdftitle={Anand Thakkar's Resume},
    pdfauthor={Anand Thakkar},
    colorlinks=true,
    urlcolor=primaryColor
}

% Formatting
\pagestyle{empty}
\setcounter{secnumdepth}{0}
\setlength{\parindent}{0pt}
\setlength{\columnsep}{0.15cm}
\pagenumbering{gobble}

\titleformat{\section}{\needspace{4\baselineskip}\bfseries\large}{}{0pt}{}[\vspace{1pt}\titlerule]

\begin{document}

% Header
\begin{center}
    {\LARGE \textbf{Anand Thakkar}}\\
    \faPhone\ +91 7778996864 \quad
    \faEnvelope\ \href{mailto:anandthakkar2208}{anandthakkar2208} \quad
    \faLinkedin\ \href{https://www.linkedin.com/in/anandthakkar-8a54s/}{LinkedIn} \quad
    \faGithub\ \href{https://github.com/astroanand-6e}{GitHub}
\end{center}

% Education
\section{Education}
\textbf{SRM Institute of Science and Technology, Kattankulathur}  \hfill \textbf{Chennai, India} \\
\textit{Bachelor of Technology in Computer Science Engineering} \hfill \textit{August 2021 - Ongoing} \\\\
\textbf{CGPA:} 7.49/10 (Till 7th Semester)

\textbf{Atmiya Vidya Mandir} \hfill \textbf{Surat} \\
\textit{12th Grade, CBSE} \hfill \textit{2021}\\
\textbf{Percentage:} 91.4\%

\textbf{Bharatiya Vidya Bhavan's Vallabhram Mehta Public School} \hfill \textbf{Vadodara} \\
\textit{10th Grade, CBSE} \hfill \textit{2019} \\
\textbf{Percentage:} 89\%

% Relevant Coursework
\section{Relevant Coursework}
\begin{itemize}[noitemsep]
    \item \textbf{Mathematics and Theory:} Calculus and Linear Algebra, Advanced Calculus and Complex Analysis, Probability and Queueing Theory, Discrete Mathematics for Engineers, Computational Logic, Transforms and Boundary Value Problems.
    \item \textbf{Core Computer Science:} Data Structures and Algorithms, Design and Analysis of Algorithms, Operating Systems, Computer Networks, Database Management Systems, Compiler Design, Software Engineering and Project Management, Formal Language and Automata.
    \item \textbf{Artificial Intelligence and Machine Learning:} Artificial Intelligence, Data Mining and Analytics, Pattern Recognition Techniques, Biometrics.
    \item \textbf{High-Performance Computing:} GPU Programming, Distributed Operating Systems.
    \item \textbf{Additional Topics:} Computer Organization and Architecture, Analog and Digital Electronics, Competitive Professional Skills (I, II, III).
\end{itemize}


% Experience
\section{Experience}

\textbf{R\&D Intern, SPAN Inspection Systems Pvt Ltd} \hfill \textit{(Dec 2024 – Present)} \\
\textit{R\&D Department — Medical Image Analysis, Deep Learning}  
\begin{itemize}
    \item Working on \textbf{Coronary Artery Segmentation and Blockage Detection and Measurement (CASBloDAM)}, improving an existing research project.  
    \item Developed a \textbf{deep learning pipeline} using \textbf{Segment Anything Model (SAM) and SAM2} for \textbf{coronary artery segmentation}.  
    \item Compared the performance of different \textbf{SAM2 models} with other research methods, identifying significant improvements.  
    \item Found that existing annotation masks were inaccurate, leading to \textbf{re-annotation of the entire dataset} and retraining models.  
    \item Achieved \textbf{81\%+ accuracy} using the \textbf{SAM2\_tiny model}, significantly outperforming previous approaches in detecting fine arteries.  
    \item Gaining expertise in \textbf{PyTorch, OpenCV, collaborative development}, and documenting projects with \textbf{Dev Notes} for maintainability.  
    \item Enhancing communication skills by collaborating within the R\&D team and presenting research findings.  
\end{itemize}

% Projects
\section{Projects}

\textbf{AI Startups \& Investors Network Visualization} 
\hfill \href{https://astroanand-6e.github.io/A_web_of_AI_startups_and_investors/}{\textbf{\textit{(Live Site)}}} \\
\begin{itemize}
    \item Developed an \textbf{interactive network graph} visualizing complex relationships between AI startups and their investors.
    \item Implemented \textbf{timeline animation} features to demonstrate the evolution of the AI investment landscape over time.
    \item Created \textbf{interactive controls} for filtering, zooming, and accessing detailed information about startups and investors.
\end{itemize}
\textit{Tech Stack:} JavaScript, D3.js v7, HTML5, CSS3, Font Awesome

\textbf{Leaf Disease Detection using Fine-Tuned SAM2\_tiny} 
\hfill \href{https://github.com/astroanand-6e/SAM2_Leaf_Disease_Segmentation}{\textbf{\textit{(GitHub Repository)}}} \\
\begin{itemize}
    \item Fine-tuned \textbf{Segment Anything 2 (SAM2\_tiny)} to enhance automatic segmentation of diseased leaf regions.
    \item Built a \textbf{deep learning pipeline} that improves semantic segmentation accuracy and robustness across plant species.
    \item Optimized model inference for real-time disease detection applications in \textbf{precision agriculture}.
\end{itemize}
\textit{Tech Stack:} Python, PyTorch, OpenCV, Pandas, NumPy

\textbf{Credit Card Approval Prediction} 
\hfill \href{https://github.com/astroanand-6e/Credit-Card-Approval-Prediction}{\textbf{\textit{(GitHub Repository)}}} \\
\begin{itemize}
    \item Engineered an advanced \textbf{stacked ensemble model} combining \textbf{Gradient Boosting, Random Forest, AdaBoost, and Neural Networks} for high-accuracy credit approval predictions.
    \item Applied feature engineering techniques like \textbf{one-hot encoding, scaling, and PCA} to optimize model performance.
    \item Evaluated model effectiveness using \textbf{ROC-AUC, precision-recall curves, and confusion matrices}.
\end{itemize}
\textit{Tech Stack:} Python, Scikit-learn, Pandas, NumPy, Matplotlib

% Publications
\section{Publications}

\textbf{ResFormerAF: Integrating Deep Learning Models for Atrial Fibrillation Detection Using ECG} \textit{(Under Review, Re-submission Planned)} \\
\begin{itemize}
    \item Spearheaded a comparative analysis of \textbf{ResNet, ResNet with attention, Bi-LSTM, and Bi-LSTM with attention} for ECG classification.
    \item Proposed a novel \textbf{ResNet + encoder-based architecture} to enhance feature extraction and classification accuracy.
    \item Achieved promising preliminary results; paper under review with plans for resubmission to a relevant AI/medical conference.
\end{itemize}

% Ongoing Research Projects
\section{Ongoing Research Projects}

\textbf{Deep Learning for Coronary Artery Stenosis Detection} \textit{(In Progress)} \\
\begin{itemize}
    \item Leading the development of a deep learning system for \textbf{coronary artery segmentation} and \textbf{stenosis detection} in medical images.
    \item Leveraging state-of-the-art models to improve diagnostic accuracy for \textbf{Coronary Artery Disease (CAD)}.
\end{itemize}

\textbf{Cross-lingual Semantic Equivalence in Large Language Models} \textit{(In Progress)} \\
\begin{itemize}
    \item Investigating how \textbf{Large Language Models (LLMs)} process semantically equivalent prompts in multilingual and code-switched contexts (e.g., Hinglish).
    \item Evaluating whether LLMs understand semantic equivalence in a human-like manner across different linguistic inputs.
    \item Initially developed for my \textbf{MATS Application (ML Alignment and Theory Scholars program)} under the mentorship of \textbf{Neel Nanda}; continuing regardless of acceptance.
\end{itemize}

% Leadership & Extracurricular Activities
\section{Leadership \& Extracurricular Activities}

\begin{itemize}
    \item \textbf{Class Representative (A2 Section, 2 Years)}  
    Successfully facilitated communication between faculty and students, ensuring smooth coordination of academic and administrative matters. Acted as a bridge between students and professors, addressing concerns, organizing discussions, and improving classroom engagement.

    \item \textbf{Founder of a GitHub Organization}  
    Leading the development of an application aimed at making learning content accessible in the language of the user's choice, including Hinglish support. Managing a team of contributors, overseeing product development, and coordinating weekly discussions to refine the app’s features.

    \item \textbf{Active Participation in Technical Workshops}  
    Attended college-organized workshops on \textbf{Game Development}, \textbf{GPU Programming}, and \textbf{GitHub Developer Tools}, enhancing skills in software optimization, interactive media creation, and collaborative coding.
\end{itemize}

% Certifications
\section{Certifications}
\textbf{Machine Learning Specialization} \\
\textit{Coursera, offered by Stanford and DeepLearning.AI} \hfill \textbf{Certificate:} \href{https://coursera.org/verify/specialization/B9AOWDR6WW1J}{\textbf{Link to Certificate}} \\
\textit{Learned modern machine learning concepts, including supervised and unsupervised learning techniques.}

\textbf{Python and Introduction to Programming} \\
\textit{Kaggle} \hfill \textbf{Certificates:} \href{https://drive.google.com/file/d/16X0GA5LLuLK1exiCj-5QL-jaMwlfS0OK/view?usp=sharing}{\textbf{Intro to Programming}} \& \href{https://drive.google.com/file/d/1Z36kwNNuXHeY8KCLyePWYNFuTbkLF2n0/view?usp=sharing}{\textbf{Python}} \\
\textit{Gained proficiency in Python programming, including data handling and algorithm design.}

\textbf{How Transformer LLMs Work!} \\
\textit{DeepLearning.AI} \hfill \textbf{Certificate:} \href{https://learn.deeplearning.ai/accomplishments/9a7b037f-773d-4353-b5c5-8302d39d94e6?usp=sharing}{\textbf{Link to Certificate}} \\
\textit{Studied the fundamentals of Transformer-based Large Language Models, including attention mechanisms and applications.}

\textbf{Deep Learning Specialization (In Progress)} \\
\textit{Coursera, offered by DeepLearning.AI} \\
\textit{Currently exploring deep learning architectures including CNNs, RNNs, and practical AI model training.}

% Research Interests / Additional Information
\section{Research Interests}
\begin{itemize}
    \item \textbf{Deep Learning Architectures:} My foundation lies in developing and understanding deep learning models, evidenced by my work with \textbf{ResNet}, \textbf{LSTM}, \textbf{Bi-LSTM}, and \textbf{UNet} architectures. This experience has provided me with a strong understanding of complex neural networks and their applications in areas like \textbf{image recognition} and \textbf{sequence modeling}.
    \item \textbf{Generative Models and Architectural Complexity:} I am particularly fascinated by the architectural innovations in generative models like \textbf{Stable Diffusion}, especially their use of \textbf{VAEs} and \textbf{UNet}. These models represent a significant leap in architectural complexity, and I am eager to contribute to their further development and understanding. This interest reflects my growing desire to explore increasingly intricate AI systems.
    \item \textbf{Human Brain \& Connectomics: The Ultimate Complex System:} My interest naturally extends to the most complex system we know – the \textbf{human brain}. Inspired by advancements in \textbf{connectomics}, such as the research highlighted by Google, I am captivated by the brain's design and architecture. I believe that insights from \textbf{neuroscience} and \textbf{connectomics} can offer invaluable inspiration for the next generation of AI architectures and vice versa. Understanding the brain's biological neural networks could lead to breakthroughs in artificial neural networks.
    \item \textbf{Mechanistic Interpretability and AI Safety:} Underpinning all my research interests is a passion for \textbf{mechanistic interpretability}. I believe that truly understanding how complex systems, whether artificial or biological, arrive at their outputs is crucial for both advancing the field and ensuring \textbf{AI safety}. My current independent research into how \textbf{LLMs process semantically equivalent prompts across languages} directly reflects this commitment to interpretability. By investigating if LLMs understand semantic equivalence in a human-like manner (as explored in \textbf{Hinglish}), I aim to contribute to a deeper understanding of their internal representations and reasoning processes.
\end{itemize}

\end{document}
